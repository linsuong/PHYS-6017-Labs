\documentclass[12pt]{article}

\usepackage{report}

\usepackage[utf8]{inputenc} % allow utf-8 input
\usepackage[T1]{fontenc}    % use 8-bit T1 fonts
\usepackage[colorlinks=true, linkcolor=black, citecolor=blue, urlcolor=blue]{hyperref}       % hyperlinks
\usepackage{url}            % simple URL typesetting
\usepackage{booktabs}       % professional-quality tables
\usepackage{amsfonts}       % blackboard math symbols
\usepackage{nicefrac}       % compact symbols for 1/2, etc.
\usepackage{microtype}      % microtypography
\usepackage{lipsum}		    
\usepackage{graphicx}
\usepackage{footnote}
\usepackage{doi}
\usepackage{comment}
\usepackage{multirow}
\usepackage{gensymb}
\usepackage{float}
\usepackage{amsmath}
\usepackage{subfig}
\usepackage[skip=10pt plus1pt, indent=30pt]{parskip}

\begin{document}

\begin{titlepage}
    \centering
    %\includegraphics[width=2.3cm]{crest.jpg}\par
    \vspace{1cm}
    {\scshape\Large Department of Physics and Astronomy \par}
    \vspace{1cm}
    {\scshape\Large The University of Southampton \par}
    \vspace{1cm}
    \vspace{1cm}
    {\huge\bfseries The Forced Simple Pendulum \par}
    \vspace{1cm}
    {\Large Ong Chin Phin (Linus) \par}
    \vspace{1cm}
    {\Large Student ID: 33184747 \par}
    \vfill
    {\large November 2023 \par}
\end{titlepage}

%\maketitle
\newpage
\tableofcontents
\thispagestyle{empty}

\newpage
\thispagestyle{empty}
\begin{abstract}
%write wha tht experiment is about and what are the results 10-15 lines

%results go here
\end{abstract}

% keywords can be removed
%\keywords{First keyword \and Second keyword \and More}

\newpage
\setcounter{page}{1}
\section{Introduction}

The equation of motion for a pendulum of a mass $m$ and length of $L$ is:
\begin{equation}
    mL^2 \frac{d^2\theta}{dt^2} + k \frac{d\theta}{dt} + mgL\sin({\theta}) = FL\cos({\Omega}t)
    \label{oscillation}
\end{equation}

Where:
\begin{itemize}
    \item $m$ is the mass of the pendulum bob
    \item $L$ is the length of the pendulum
    \item $\theta$ is the angular displacement of the pendulum from vertical
    \item $\dot{\theta}$ is the angular velocity (rate of change of $\theta$)
    \item $\ddot{\theta}$ is the angular acceleration
    \item $k$ is the damping coefficient (dependent on the specific damping mechanism)
    \item $g$ is acceleration due to gravity
    \item $F$ is the magnitude of the external force
    \item $\Omega$ is the frequency of the external force $F$
    \item $t$ is the time
\end{itemize}

It is also useful to note that $k\dot{\theta}$ is the damping force.
We measure time relative to the period of free oscillations of small amplitude, so we can introduce:
\begin{equation}
    t = \tau\sqrt{\frac{L}{g}}
    \label{modified time}
\end{equation}

We introduce:
\begin{equation}
    \Omega = (1 - \eta)\sqrt{\frac{g}{L}}
    \label{modified omega}
\end{equation}

To investigate the situation where the behavior of the pendulum when the forcing frequency, $\Omega$ is slightly less than the natural frequency, $\frac{g}{L}$.

To convert equation (\ref{oscillation}) into dimensionless form, we recall that: $t = \tau\sqrt{\frac{L}{g}}$ (equation (\ref{modified time})) and by the chain rule, $\frac{d\tau}{dt} = \frac{d\theta}{dt}\frac{dt}{d\tau}$ give the following equations:
\begin{equation}
    \begin{split}
        \frac{d\theta}{dt} = \sqrt{\frac{g}{L}}\frac{d\theta}{d\tau} \\
        \frac{d^2\theta}{dt^2} = \frac{g}{L}\frac{d^2\theta}{d\tau^2}
    \end{split}
    \label{chain rules}
\end{equation}

Which can be substituted into equation (\ref{oscillation}) after dividing by $mgL$ and substituting equation (\ref*{modified time}) and (\ref{modified omega}).
\begin{equation}
    (\frac{L}{g})(\frac{g}{L})\frac{d^2\theta}{d\tau^2} + \frac{k}{mgL}\sqrt{\frac{g}{L}}\frac{d\theta}{d\tau} + \sin(\theta) = \frac{F}{mg}\cos((1 - \eta)\sqrt{\frac{g}{L}}\sqrt{\frac{L}{g}}\tau)
\end{equation}

The dimensionless form of equation (\ref{oscillation}) is then:
\begin{equation}
    \frac{d^2\theta}{d\tau^2} = -\alpha\frac{d\theta}{d\tau} - \sin(\theta) + \beta\cos((1 - \eta)\tau)
    \label{dimensionless}
\end{equation}

With parameters: 
\begin{equation}
    \centering
    \begin{split}
        \alpha &= \frac{k}{mL\sqrt{gL}}\\
        \beta &= \frac{F}{mg}
    \end{split}
    \label{parameters}
\end{equation}

\footnote{
    Note that:
    \\
        \begin{equation}
            \frac{k}{mgL}\sqrt{\frac{g}{L}} = \frac{k}{mL\sqrt{g}\cdot\sqrt{g}}\sqrt{\frac{g}{L}} = \frac{k}{mL\sqrt{gL}}
        \end{equation}
}

To solve the equation, the RK4(5) (Runge–Kutta–Fehlberg) method will be used. I will be using \verb|scipy.integrate.RK45| that uses the Dormand-Prince pair of formulas \cite{DormandPrince}. To solve a 2nd order differential equation with any Runge-Kutta method, the ODE in question will have to be expressed as a 1st order ODE. 
From equation (\ref{dimensionless}) and taking the substitution:
\begin{equation}
    \omega = \frac{d\theta}{d\tau}
\end{equation}

The ODE will be a system of equations of the form $\vec{\bold{Y'}} = A\vec{\bold{Y}} - \vec{\bold{b}}$.
\begin{gather}
    \begin{bmatrix}
        \theta' \\ 
        \omega'
    \end{bmatrix}
    =
    \begin{bmatrix}
        0 & 1 \\
        $-\sin(\theta)$ & $-\alpha$
    \end{bmatrix}
    \begin{bmatrix}
        \theta \\ 
        \omega
    \end{bmatrix}
    -
    \begin{bmatrix}
        0 \\
        $\beta\cos((1 - \eta)\tau)$        
    \end{bmatrix}
\end{gather}

\section{Setup}
The system is set up as follows:
\begin{itemize}
    \item The pendulum has a mass $m$ of 1kg
    \item The length of the string, $L$ is 2 meters
    \item The damping coefficient $k$ has a value of 
    \item The factor $\eta$ has a value of 0.01
    \item $g$ is 9.81$ms^{-1}$
\end{itemize}

\subsection{Areas of investigation}
\begin{figure}
    \centering
    \includegraphics{}
    \caption{Caption}
    \label{fig:enter-label}
\end{figure}

\section{Addendum}
%stuff here
\subsection{Random Walks}
Random walks are a way to 
\subsubsection{In Academia}
\subsubsection{Outside of Academia}
\subsection{Monte Carlo Simulations}
The famous Monte Carlo simulation is a way to solve complex problems involving many variables, which may
\subsubsection{In Academia}
\subsubsection{Outside of Academia}


\subsection{Random Numbers}
\subsubsection{In Academia}
\subsubsection{Outside of Academia}

\newpage
\section{Bibliography}
\bibliographystyle{plain}
\bibliography{references}

\end{document}